\documentclass[main.tex]{subfiles}
\begin{document}
\section{有感FOC算法代码详解}
\subsection{Clark变换}
\begin{lstlisting}[language=C]
void Clark_Transform(FOC_Typedef_t *foc)
{
    foc->Curr.Ia = adc.Current[0];
    foc->Curr.Ib = adc.Current[1];
    foc->Curr.Ic = adc.Current[2];
    // Ia + Ib + Ic = 0
    foc->Curr.Ialpha = foc->Curr.Ia;
    foc->Curr.Ibeta = SQRT3_3 * (foc->Curr.Ia + 2.0f * foc->Curr.Ib);
}
\end{lstlisting}

\subsection{Park变换}
\begin{lstlisting}[language=C]
void Park_Transform(FOC_Typedef_t *foc)
{
    foc->Curr.Id = foc->Curr.Ialpha * foc->cos_theta 
                 + foc->Curr.Ibeta * foc->sin_theta;
    foc->Curr.Iq = -foc->Curr.Ialpha * foc->sin_theta 
                 + foc->Curr.Ibeta * foc->cos_theta;
}
\end{lstlisting}
\subsection{电流闭环}
\begin{lstlisting}[language=C]
void PID_Current_Loop_Calc(FOC_Typedef_t *foc)
{
	PID_Calc(&Id_Pid,foc->Curr.Id,foc->Curr.Id_set);
	PID_Calc(&Iq_Pid,foc->Curr.Iq,foc->Curr.Iq_set);
	foc->Volt.Vd = Id_Pid.out;
	foc->Volt.Vq = Iq_Pid.out;
}
\end{lstlisting}
PID基础及PID计算函数略过不讲。

\subsection{逆Park变换}
\begin{lstlisting}[language=C]
void InvPark_Transform(FOC_Typedef_t *foc)
{
    foc->Volt.Valpha = foc->Volt.Vd * foc->cos_theta 
                     - foc->Volt.Vq * foc->sin_theta;
    foc->Volt.Vbeta = foc->Volt.Vd * foc->sin_theta 
                    + foc->Volt.Vq * foc->cos_theta;
}
\end{lstlisting}

\subsection{SVPWM}
\begin{lstlisting}[language=C]
void SVPWM_Calc(FOC_Typedef_t *foc)
{
	uint8_t A, B, C, N;
	fp32 Uref1, Uref2, Uref3;
	fp32 X, Y, Z;
	int16_t T0, T1, T2;
	int16_t Ta, Tb, Tc;
		
	Uref1 = foc->Volt.Vbeta;
	Uref2 = SQRT3_2 * foc->Volt.Valpha - 0.5f * foc->Volt.Vbeta;
	Uref3 = - SQRT3_2 * foc->Volt.Valpha - 0.5f * foc->Volt.Vbeta;
	// A, B and C are the temp flag to calculate the sector of the motor
	if (Uref1 > 0.0f) A = 1; else A = 0;
	if (Uref2 > 0.0f) B = 1; else B = 0;
	if (Uref3 > 0.0f) C = 1; else C = 0;
	// N is one-to-one mapping of the sector
	N = 4 * C + 2 * B + A;
	// X, Y and Z can help to get T0, T4 and T6
	X = SQRT3 * foc->Volt.Vbeta * PWM_TS / adc.Vbus;
	Y = 1.5f * foc->Volt.Valpha * PWM_TS / adc.Vbus 
          + SQRT3_2 * foc->Volt.Vbeta * PWM_TS / adc.Vbus;
	Z = -1.5f * foc->Volt.Valpha * PWM_TS / adc.Vbus 
          + SQRT3_2 * foc->Volt.Vbeta * PWM_TS / adc.Vbus;
	// T1 is T4, and T2 is T6, actually the SECTOR1 to SECTOR2 is not 
        // the real sector, they are also one-to-one mapping of the sector
	switch (N) {
		case SECTOR1: {
		  T1 = Z;
		  T2 = Y;
		  break;
		}
		case SECTOR2: {
		  T1 = Y;
		  T2 = -X;
		  break;
		}
		case SECTOR3: {
		  T1 = -Z;
		  T2 = X;
		  break;
		}
		case SECTOR4: {
		  T1 = -X;
                  T2 = Z;
		  break;
		}
		case SECTOR5: {
		  T1 = X;
                  T2 = -Y;
		  break;
		}
		case SECTOR6: {
		  T1 = -Y;
                  T2 = -Z;
		  break;
		}
		default: {
		  T1 = 0;
                  T2 = 0;
                  break;
		}
	}
	// The T1, T2 and T0 can not over the maximum PWM shadow register
	PWM_Output_Limit(T1);
	PWM_Output_Limit(T2);
	Over_Modulation(T1, T2);
	T0 = (PWM_TS - T1 - T2) / 2;
        PWM_Output_Limit(T0);
	// Seven-segment SVPWM time switching point calculation, and put the 
	// answers to CCR registers
	Ta = (PWM_TS - T1 - T2)/4;
	Tb = Ta + T1/2;
	Tc = Tb + T2/2;
	switch(N)
	{
		case SECTOR1: {
			TIM1->CCR1 = (Tb);
			TIM1->CCR2 = (Ta);
			TIM1->CCR3 = (Tc);
			break;
		}
		case SECTOR2: {
			TIM1->CCR1 = (Ta);
                        TIM1->CCR2 = (Tc);
			TIM1->CCR3 = (Tb);
                        break;
		}
		case SECTOR3: {
			TIM1->CCR1 = (Ta);
			TIM1->CCR2 = (Tb);
			TIM1->CCR3 = (Tc);
			break;
		}
		case SECTOR4: {
			TIM1->CCR1 = (Tc);
                        TIM1->CCR2 = (Tb);
			TIM1->CCR3 = (Ta);
			break;
		}
		case SECTOR5: {
			TIM1->CCR1 = (Tc);
			TIM1->CCR2 = (Ta);
			TIM1->CCR3 = (Tb);
			break;
		}
		case SECTOR6: {
			TIM1->CCR1 = (Tb);
                        TIM1->CCR2 = (Tc);
                        TIM1->CCR3 = (Ta);
                        break;
		}
		default: {
			TIM1->CCR1 = PWM_TS * 0.5f;
			TIM1->CCR2 = PWM_TS * 0.5f;
			TIM1->CCR3 = PWM_TS * 0.5f;
			break;
		}
	}
}
\end{lstlisting}
\end{document}
